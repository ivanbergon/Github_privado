% Options for packages loaded elsewhere
\PassOptionsToPackage{unicode}{hyperref}
\PassOptionsToPackage{hyphens}{url}
%
\documentclass[
]{article}
\author{}
\date{\vspace{-2.5em}}

\usepackage{amsmath,amssymb}
\usepackage{lmodern}
\usepackage{iftex}
\ifPDFTeX
  \usepackage[T1]{fontenc}
  \usepackage[utf8]{inputenc}
  \usepackage{textcomp} % provide euro and other symbols
\else % if luatex or xetex
  \usepackage{unicode-math}
  \defaultfontfeatures{Scale=MatchLowercase}
  \defaultfontfeatures[\rmfamily]{Ligatures=TeX,Scale=1}
\fi
% Use upquote if available, for straight quotes in verbatim environments
\IfFileExists{upquote.sty}{\usepackage{upquote}}{}
\IfFileExists{microtype.sty}{% use microtype if available
  \usepackage[]{microtype}
  \UseMicrotypeSet[protrusion]{basicmath} % disable protrusion for tt fonts
}{}
\makeatletter
\@ifundefined{KOMAClassName}{% if non-KOMA class
  \IfFileExists{parskip.sty}{%
    \usepackage{parskip}
  }{% else
    \setlength{\parindent}{0pt}
    \setlength{\parskip}{6pt plus 2pt minus 1pt}}
}{% if KOMA class
  \KOMAoptions{parskip=half}}
\makeatother
\usepackage{xcolor}
\IfFileExists{xurl.sty}{\usepackage{xurl}}{} % add URL line breaks if available
\IfFileExists{bookmark.sty}{\usepackage{bookmark}}{\usepackage{hyperref}}
\hypersetup{
  hidelinks,
  pdfcreator={LaTeX via pandoc}}
\urlstyle{same} % disable monospaced font for URLs
\usepackage[margin=1in]{geometry}
\usepackage{color}
\usepackage{fancyvrb}
\newcommand{\VerbBar}{|}
\newcommand{\VERB}{\Verb[commandchars=\\\{\}]}
\DefineVerbatimEnvironment{Highlighting}{Verbatim}{commandchars=\\\{\}}
% Add ',fontsize=\small' for more characters per line
\usepackage{framed}
\definecolor{shadecolor}{RGB}{248,248,248}
\newenvironment{Shaded}{\begin{snugshade}}{\end{snugshade}}
\newcommand{\AlertTok}[1]{\textcolor[rgb]{0.94,0.16,0.16}{#1}}
\newcommand{\AnnotationTok}[1]{\textcolor[rgb]{0.56,0.35,0.01}{\textbf{\textit{#1}}}}
\newcommand{\AttributeTok}[1]{\textcolor[rgb]{0.77,0.63,0.00}{#1}}
\newcommand{\BaseNTok}[1]{\textcolor[rgb]{0.00,0.00,0.81}{#1}}
\newcommand{\BuiltInTok}[1]{#1}
\newcommand{\CharTok}[1]{\textcolor[rgb]{0.31,0.60,0.02}{#1}}
\newcommand{\CommentTok}[1]{\textcolor[rgb]{0.56,0.35,0.01}{\textit{#1}}}
\newcommand{\CommentVarTok}[1]{\textcolor[rgb]{0.56,0.35,0.01}{\textbf{\textit{#1}}}}
\newcommand{\ConstantTok}[1]{\textcolor[rgb]{0.00,0.00,0.00}{#1}}
\newcommand{\ControlFlowTok}[1]{\textcolor[rgb]{0.13,0.29,0.53}{\textbf{#1}}}
\newcommand{\DataTypeTok}[1]{\textcolor[rgb]{0.13,0.29,0.53}{#1}}
\newcommand{\DecValTok}[1]{\textcolor[rgb]{0.00,0.00,0.81}{#1}}
\newcommand{\DocumentationTok}[1]{\textcolor[rgb]{0.56,0.35,0.01}{\textbf{\textit{#1}}}}
\newcommand{\ErrorTok}[1]{\textcolor[rgb]{0.64,0.00,0.00}{\textbf{#1}}}
\newcommand{\ExtensionTok}[1]{#1}
\newcommand{\FloatTok}[1]{\textcolor[rgb]{0.00,0.00,0.81}{#1}}
\newcommand{\FunctionTok}[1]{\textcolor[rgb]{0.00,0.00,0.00}{#1}}
\newcommand{\ImportTok}[1]{#1}
\newcommand{\InformationTok}[1]{\textcolor[rgb]{0.56,0.35,0.01}{\textbf{\textit{#1}}}}
\newcommand{\KeywordTok}[1]{\textcolor[rgb]{0.13,0.29,0.53}{\textbf{#1}}}
\newcommand{\NormalTok}[1]{#1}
\newcommand{\OperatorTok}[1]{\textcolor[rgb]{0.81,0.36,0.00}{\textbf{#1}}}
\newcommand{\OtherTok}[1]{\textcolor[rgb]{0.56,0.35,0.01}{#1}}
\newcommand{\PreprocessorTok}[1]{\textcolor[rgb]{0.56,0.35,0.01}{\textit{#1}}}
\newcommand{\RegionMarkerTok}[1]{#1}
\newcommand{\SpecialCharTok}[1]{\textcolor[rgb]{0.00,0.00,0.00}{#1}}
\newcommand{\SpecialStringTok}[1]{\textcolor[rgb]{0.31,0.60,0.02}{#1}}
\newcommand{\StringTok}[1]{\textcolor[rgb]{0.31,0.60,0.02}{#1}}
\newcommand{\VariableTok}[1]{\textcolor[rgb]{0.00,0.00,0.00}{#1}}
\newcommand{\VerbatimStringTok}[1]{\textcolor[rgb]{0.31,0.60,0.02}{#1}}
\newcommand{\WarningTok}[1]{\textcolor[rgb]{0.56,0.35,0.01}{\textbf{\textit{#1}}}}
\usepackage{longtable,booktabs,array}
\usepackage{calc} % for calculating minipage widths
% Correct order of tables after \paragraph or \subparagraph
\usepackage{etoolbox}
\makeatletter
\patchcmd\longtable{\par}{\if@noskipsec\mbox{}\fi\par}{}{}
\makeatother
% Allow footnotes in longtable head/foot
\IfFileExists{footnotehyper.sty}{\usepackage{footnotehyper}}{\usepackage{footnote}}
\makesavenoteenv{longtable}
\usepackage{graphicx}
\makeatletter
\def\maxwidth{\ifdim\Gin@nat@width>\linewidth\linewidth\else\Gin@nat@width\fi}
\def\maxheight{\ifdim\Gin@nat@height>\textheight\textheight\else\Gin@nat@height\fi}
\makeatother
% Scale images if necessary, so that they will not overflow the page
% margins by default, and it is still possible to overwrite the defaults
% using explicit options in \includegraphics[width, height, ...]{}
\setkeys{Gin}{width=\maxwidth,height=\maxheight,keepaspectratio}
% Set default figure placement to htbp
\makeatletter
\def\fps@figure{htbp}
\makeatother
\setlength{\emergencystretch}{3em} % prevent overfull lines
\providecommand{\tightlist}{%
  \setlength{\itemsep}{0pt}\setlength{\parskip}{0pt}}
\setcounter{secnumdepth}{-\maxdimen} % remove section numbering
\ifLuaTeX
  \usepackage{selnolig}  % disable illegal ligatures
\fi

\begin{document}

\hypertarget{argus-media---oil-dataset-project}{%
\section{Argus Media - Oil Dataset
Project}\label{argus-media---oil-dataset-project}}

\hypertarget{ivan-berlim-gonuxe7alves}{%
\subsection{Ivan Berlim Gonçalves}\label{ivan-berlim-gonuxe7alves}}

\begin{center}\rule{0.5\linewidth}{0.5pt}\end{center}

\hypertarget{business-problem}{%
\subsubsection{1 - BUSINESS PROBLEM}\label{business-problem}}

We want to predict the \texttt{OILPRICE} values given the oil dataset
from GAMLSS package.

The price of oil and other commodities is influenced by three major
factors: supply, demand and geopolitics. But what if we could indirectly
predict its price using other commodities and stock indexes? This is the
proposed challenge that we are going to solve. We are going to make a
superficial scratch on the complex world of natural resource pricing.

Great! We already have our problem question:

\begin{itemize}
\tightlist
\item
  What will be the tomorrow's oil price?
\end{itemize}

Now let us establish some tangible objectives to what we want as an
answer. Given that the stock market is a complex, volatile and fast
environment, we must set a goal for assessing the model's metrics. A
variation of 2\% in price prediction could mean a lot of wasted money
depending on the amount applied. So we are going to set two high
standards goals for our model performance:

\begin{itemize}
\tightlist
\item
  at least 90\% of the predictions have to be on a 1\% deviance margin
  from the real values.
\item
  at least 75\% of the predictions have to be on a 0.5\% deviance margin
  from the real values.
\end{itemize}

We are ready to start. It's time to understand the data.

\begin{center}\rule{0.5\linewidth}{0.5pt}\end{center}

\hypertarget{understanding-the-data}{%
\subsubsection{2 - UNDERSTANDING THE
DATA}\label{understanding-the-data}}

\hypertarget{what-is-gamlss}{%
\paragraph{What is GAMLSS?}\label{what-is-gamlss}}

Generalized Additive Models for Location, Scale and Shape (GAMLSS) were
introduced by Rigby and Stasinopoulos (2001, 2005) and Akantziliotou et
al.~(2002) as a way of overcoming some of the limitations associated
with Generalized Linear Models (GLM) and Generalized Additive Models
(GAM) (Nelder and Wedderburn, 1972 and Hastie and Tibshirani, 1990,
respectively).

In GAMLSS the exponential family distribution assumption for the
response variable (Y) is relaxed and replaced by a general distribution
family, including highly skew and/or kurtotic distributions. The
systematic part of the model is expanded to allow modeling not only the
mean (or location) but other parameters of the distribution of Y as
linear parametric and/or additive non-parametric functions of
explanatory variables and/or random effects. Maximum (penalized)
likelihood estimation is used to fit the models.

There are two algorithms to fit the models, the CG and RS algorithms.

\begin{center}\rule{0.5\linewidth}{0.5pt}\end{center}

\hypertarget{gamlss-guide-used-to-this-project}{%
\paragraph{GAMLSS guide used to this
project:}\label{gamlss-guide-used-to-this-project}}

\url{http://www.gamlss.com/wp-content/uploads/2013/01/gamlss-manual.pdf}

\begin{center}\rule{0.5\linewidth}{0.5pt}\end{center}

\hypertarget{dataset-information}{%
\paragraph{Dataset Information:}\label{dataset-information}}

Link: \url{https://rdrr.io/cran/gamlss.data/man/oil.html}

\begin{center}\rule{0.5\linewidth}{0.5pt}\end{center}

\hypertarget{dataset-description}{%
\paragraph{Dataset Description}\label{dataset-description}}

The Oil data: Using model selection to discover what affects the price
of oil. The data s contains the daily prices of front-month WTI (West
Texas Intermediate) oil prices traded by NYMEX (New York Mercantile
Exchange). The front-month WTI oil price is a futures contract with the
shortest duration that could be purchased in the NYMEX market. The idea
is to use other financially traded products (e.g., gold price) to
discover what might affect the daily dynamics of the price of oil.

\begin{center}\rule{0.5\linewidth}{0.5pt}\end{center}

\hypertarget{dataset-source}{%
\paragraph{Dataset Source}\label{dataset-source}}

The dataset was downloaded from \url{https://www.quandl.com/}.

\begin{center}\rule{0.5\linewidth}{0.5pt}\end{center}

\hypertarget{dataset-format}{%
\paragraph{Dataset Format}\label{dataset-format}}

\hypertarget{oilprice}{%
\subparagraph{OILPRICE}\label{oilprice}}

the log price of front month WTI oil contract traded by NYMEX - in
financial terms, this is the CL1. This is the response variable.

\hypertarget{cl_log}{%
\subparagraph{CL\_log}\label{cl_log}}

numeric vectors which are the log prices of the 2 to 15 months ahead WTI
oil contracts traded by NYMEX. For example, for the trading day of 2nd
June 2016, the CL2 is the WTI oil contract for delivery in August 2016.

\hypertarget{bdiy_log}{%
\subparagraph{BDIY\_log}\label{bdiy_log}}

the Baltic Dry Index, which is an assessment of the price of moving the
major raw materials by sea.

\hypertarget{spx_log}{%
\subparagraph{SPX\_log}\label{spx_log}}

the S\&P 500 index

\hypertarget{dx1_log}{%
\subparagraph{DX1\_log}\label{dx1_log}}

the US Dollar Index

\hypertarget{gc1_log}{%
\subparagraph{GC1\_log}\label{gc1_log}}

The log price of front month gold price contract traded by NYMEX

\hypertarget{ho1_log}{%
\subparagraph{HO1\_log}\label{ho1_log}}

the log price of front month heating oil contract traded by NYMEX

\hypertarget{usci_log}{%
\subparagraph{USCI\_log}\label{usci_log}}

the United States Commodity Index

\hypertarget{gnr_log}{%
\subparagraph{GNR\_log}\label{gnr_log}}

the S\&P Global Natural Resources Index

\hypertarget{shcomp_log}{%
\subparagraph{SHCOMP\_log}\label{shcomp_log}}

the Shanghai Stock Exchange Composite Index.

\hypertarget{ftse_log}{%
\subparagraph{FTSE\_log}\label{ftse_log}}

the FTSE 100 Index

\hypertarget{resplag}{%
\subparagraph{respLAG}\label{resplag}}

the lag 1 of OILPRICE - lagged version of the response variable.

\begin{center}\rule{0.5\linewidth}{0.5pt}\end{center}

\hypertarget{data-preparation}{%
\subsubsection{3 - DATA PREPARATION}\label{data-preparation}}

\begin{center}\rule{0.5\linewidth}{0.5pt}\end{center}

\hypertarget{packages-and-libraries}{%
\paragraph{PACKAGES and LIBRARIES}\label{packages-and-libraries}}

\begin{Shaded}
\begin{Highlighting}[]
\CommentTok{\# Verify if the package is already installed, if not, install package}
\ControlFlowTok{if}\NormalTok{(}\StringTok{"ggplot2"} \SpecialCharTok{\%in\%} \FunctionTok{rownames}\NormalTok{(}\FunctionTok{installed.packages}\NormalTok{()) }\SpecialCharTok{==} \ConstantTok{FALSE}\NormalTok{) \{}\FunctionTok{install.packages}\NormalTok{(}\StringTok{"ggplot2"}\NormalTok{)\}}
\ControlFlowTok{if}\NormalTok{(}\StringTok{"ggpubr"} \SpecialCharTok{\%in\%} \FunctionTok{rownames}\NormalTok{(}\FunctionTok{installed.packages}\NormalTok{()) }\SpecialCharTok{==} \ConstantTok{FALSE}\NormalTok{) \{}\FunctionTok{install.packages}\NormalTok{(}\StringTok{"ggpubr"}\NormalTok{)\}}
\ControlFlowTok{if}\NormalTok{(}\StringTok{"tidyverse"} \SpecialCharTok{\%in\%} \FunctionTok{rownames}\NormalTok{(}\FunctionTok{installed.packages}\NormalTok{()) }\SpecialCharTok{==} \ConstantTok{FALSE}\NormalTok{) \{}\FunctionTok{install.packages}\NormalTok{(}\StringTok{"tidyverse"}\NormalTok{)\}}
\ControlFlowTok{if}\NormalTok{(}\StringTok{"tidyr"} \SpecialCharTok{\%in\%} \FunctionTok{rownames}\NormalTok{(}\FunctionTok{installed.packages}\NormalTok{()) }\SpecialCharTok{==} \ConstantTok{FALSE}\NormalTok{) \{}\FunctionTok{install.packages}\NormalTok{(}\StringTok{"tidyr"}\NormalTok{)\}}
\ControlFlowTok{if}\NormalTok{(}\StringTok{"gridExtra"} \SpecialCharTok{\%in\%} \FunctionTok{rownames}\NormalTok{(}\FunctionTok{installed.packages}\NormalTok{()) }\SpecialCharTok{==} \ConstantTok{FALSE}\NormalTok{) \{}\FunctionTok{install.packages}\NormalTok{(}\StringTok{"gridExtra"}\NormalTok{)\}}
\ControlFlowTok{if}\NormalTok{(}\StringTok{"gamlss"} \SpecialCharTok{\%in\%} \FunctionTok{rownames}\NormalTok{(}\FunctionTok{installed.packages}\NormalTok{()) }\SpecialCharTok{==} \ConstantTok{FALSE}\NormalTok{) \{}\FunctionTok{install.packages}\NormalTok{(}\StringTok{"gamlss"}\NormalTok{)\}}
\ControlFlowTok{if}\NormalTok{(}\StringTok{"gamlss.add"} \SpecialCharTok{\%in\%} \FunctionTok{rownames}\NormalTok{(}\FunctionTok{installed.packages}\NormalTok{()) }\SpecialCharTok{==} \ConstantTok{FALSE}\NormalTok{) \{}\FunctionTok{install.packages}\NormalTok{(}\StringTok{"gamlss.add"}\NormalTok{)\}}
\ControlFlowTok{if}\NormalTok{(}\StringTok{"gamlss.dist"} \SpecialCharTok{\%in\%} \FunctionTok{rownames}\NormalTok{(}\FunctionTok{installed.packages}\NormalTok{()) }\SpecialCharTok{==} \ConstantTok{FALSE}\NormalTok{) \{}\FunctionTok{install.packages}\NormalTok{(}\StringTok{"gamlss.dist"}\NormalTok{)\}}
\ControlFlowTok{if}\NormalTok{(}\StringTok{"corrplot"} \SpecialCharTok{\%in\%} \FunctionTok{rownames}\NormalTok{(}\FunctionTok{installed.packages}\NormalTok{()) }\SpecialCharTok{==} \ConstantTok{FALSE}\NormalTok{) \{}\FunctionTok{install.packages}\NormalTok{(}\StringTok{"corrplot"}\NormalTok{)\}}
\ControlFlowTok{if}\NormalTok{(}\StringTok{"gamlss.dist"} \SpecialCharTok{\%in\%} \FunctionTok{rownames}\NormalTok{(}\FunctionTok{installed.packages}\NormalTok{()) }\SpecialCharTok{==} \ConstantTok{FALSE}\NormalTok{) \{}\FunctionTok{install.packages}\NormalTok{(}\StringTok{"gamlss.dist"}\NormalTok{)\}}
\ControlFlowTok{if}\NormalTok{(}\StringTok{"Hmisc"} \SpecialCharTok{\%in\%} \FunctionTok{rownames}\NormalTok{(}\FunctionTok{installed.packages}\NormalTok{()) }\SpecialCharTok{==} \ConstantTok{FALSE}\NormalTok{) \{}\FunctionTok{install.packages}\NormalTok{(}\StringTok{"Hmisc"}\NormalTok{)\}}

\CommentTok{\# Loading libraries}
\FunctionTok{library}\NormalTok{(gamlss)}
\end{Highlighting}
\end{Shaded}

\begin{verbatim}
## Carregando pacotes exigidos: splines
\end{verbatim}

\begin{verbatim}
## Carregando pacotes exigidos: gamlss.data
\end{verbatim}

\begin{verbatim}
## 
## Attaching package: 'gamlss.data'
\end{verbatim}

\begin{verbatim}
## The following object is masked from 'package:datasets':
## 
##     sleep
\end{verbatim}

\begin{verbatim}
## Carregando pacotes exigidos: gamlss.dist
\end{verbatim}

\begin{verbatim}
## Carregando pacotes exigidos: MASS
\end{verbatim}

\begin{verbatim}
## Carregando pacotes exigidos: nlme
\end{verbatim}

\begin{verbatim}
## Carregando pacotes exigidos: parallel
\end{verbatim}

\begin{verbatim}
##  **********   GAMLSS Version 5.3-4  **********
\end{verbatim}

\begin{verbatim}
## For more on GAMLSS look at https://www.gamlss.com/
\end{verbatim}

\begin{verbatim}
## Type gamlssNews() to see new features/changes/bug fixes.
\end{verbatim}

\begin{Shaded}
\begin{Highlighting}[]
\FunctionTok{library}\NormalTok{(ggpubr)}
\end{Highlighting}
\end{Shaded}

\begin{verbatim}
## Carregando pacotes exigidos: ggplot2
\end{verbatim}

\begin{Shaded}
\begin{Highlighting}[]
\FunctionTok{library}\NormalTok{(gamlss.add)}
\end{Highlighting}
\end{Shaded}

\begin{verbatim}
## Carregando pacotes exigidos: mgcv
\end{verbatim}

\begin{verbatim}
## This is mgcv 1.8-36. For overview type 'help("mgcv-package")'.
\end{verbatim}

\begin{verbatim}
## Carregando pacotes exigidos: nnet
\end{verbatim}

\begin{verbatim}
## 
## Attaching package: 'nnet'
\end{verbatim}

\begin{verbatim}
## The following object is masked from 'package:mgcv':
## 
##     multinom
\end{verbatim}

\begin{verbatim}
## Carregando pacotes exigidos: rpart
\end{verbatim}

\begin{Shaded}
\begin{Highlighting}[]
\FunctionTok{library}\NormalTok{(gamlss.dist)}
\FunctionTok{library}\NormalTok{(ggplot2)}
\FunctionTok{library}\NormalTok{(corrplot)}
\end{Highlighting}
\end{Shaded}

\begin{verbatim}
## corrplot 0.90 loaded
\end{verbatim}

\begin{Shaded}
\begin{Highlighting}[]
\FunctionTok{library}\NormalTok{(cowplot)}
\end{Highlighting}
\end{Shaded}

\begin{verbatim}
## 
## Attaching package: 'cowplot'
\end{verbatim}

\begin{verbatim}
## The following object is masked from 'package:ggpubr':
## 
##     get_legend
\end{verbatim}

\begin{Shaded}
\begin{Highlighting}[]
\FunctionTok{library}\NormalTok{(Hmisc)}
\end{Highlighting}
\end{Shaded}

\begin{verbatim}
## Carregando pacotes exigidos: lattice
\end{verbatim}

\begin{verbatim}
## Carregando pacotes exigidos: survival
\end{verbatim}

\begin{verbatim}
## Carregando pacotes exigidos: Formula
\end{verbatim}

\begin{verbatim}
## 
## Attaching package: 'Hmisc'
\end{verbatim}

\begin{verbatim}
## The following objects are masked from 'package:base':
## 
##     format.pval, units
\end{verbatim}

\begin{Shaded}
\begin{Highlighting}[]
\FunctionTok{library}\NormalTok{(gridExtra)}
\end{Highlighting}
\end{Shaded}

\begin{center}\rule{0.5\linewidth}{0.5pt}\end{center}

Extract the GAMLSS Oil dataset as \texttt{oil} data frame:

\begin{Shaded}
\begin{Highlighting}[]
\CommentTok{\# extrack the \textquotesingle{}oil\textquotesingle{} data}
\FunctionTok{data}\NormalTok{(oil)}

\CommentTok{\# set random seed to create a reproducible script}
\FunctionTok{set.seed}\NormalTok{(}\DecValTok{18}\NormalTok{)}
\end{Highlighting}
\end{Shaded}

\begin{center}\rule{0.5\linewidth}{0.5pt}\end{center}

\hypertarget{exploratory-analysis}{%
\subsubsection{4 - EXPLORATORY ANALYSIS}\label{exploratory-analysis}}

We want to understand how the data is organized in the data frame:

\begin{Shaded}
\begin{Highlighting}[]
\CommentTok{\# Head function to see the \textquotesingle{}oil\textquotesingle{} data frame}
\FunctionTok{head}\NormalTok{(oil)}
\end{Highlighting}
\end{Shaded}

\begin{verbatim}
##   OILPRICE  CL2_log  CL3_log  CL4_log  CL5_log  CL6_log  CL7_log  CL8_log
## 1 4.640923 4.636475 4.641116 4.644968 4.648038 4.649761 4.650908 4.651863
## 2 4.633077 4.645352 4.649857 4.653484 4.656338 4.657858 4.658711 4.659564
## 3 4.634049 4.637831 4.642466 4.646312 4.649665 4.651672 4.653103 4.654341
## 4 4.646312 4.638315 4.642562 4.646120 4.648708 4.649952 4.650621 4.651099
## 5 4.631520 4.650526 4.654722 4.658047 4.660321 4.661267 4.661740 4.662117
## 6 4.627616 4.635893 4.640344 4.644199 4.646984 4.648421 4.649378 4.650335
##    CL9_log CL10_log CL11_log CL12_log CL13_log CL14_log CL15_log BDIY_log
## 1 4.652340 4.651672 4.650621 4.648613 4.646120 4.643236 4.639765 6.850126
## 2 4.660037 4.659374 4.657952 4.655578 4.652531 4.649187 4.645256 6.850126
## 3 4.655293 4.655007 4.653865 4.651672 4.648804 4.645736 4.642081 6.879356
## 4 4.651577 4.651004 4.649570 4.647080 4.644102 4.640923 4.636960 6.882437
## 5 4.662401 4.661645 4.659942 4.656908 4.653484 4.649761 4.645544 6.896694
## 6 4.651099 4.650813 4.649570 4.646984 4.643910 4.640537 4.636475 6.913737
##    SPX_log  DX1_log  GC1_log  HO1_log USCI_log  GNR_log SHCOMP_log FTSE_log
## 1 7.221624 4.386554 7.413367 1.136197 4.108412 3.917806   7.744539 8.636699
## 2 7.235309 4.379762 7.419680 1.152564 4.120986 3.942552   7.762536 8.650062
## 3 7.222756 4.387449 7.418481 1.155182 4.115127 3.923952   7.766061 8.639729
## 4 7.222252 4.383675 7.410347 1.136743 4.103965 3.925531   7.765158 8.642292
## 5 7.237620 4.382364 7.399704 1.139946 4.107096 3.941970   7.755763 8.659907
## 6 7.233556 4.382951 7.404888 1.137256 4.097672 3.936325   7.775213 8.656137
##    respLAG
## 1 4.631812
## 2 4.640923
## 3 4.633077
## 4 4.634049
## 5 4.646312
## 6 4.631520
\end{verbatim}

\begin{center}\rule{0.5\linewidth}{0.5pt}\end{center}

Let's see how many rows (observations) and columns (variables) there are
in the \texttt{oil} data frame:

\begin{Shaded}
\begin{Highlighting}[]
\FunctionTok{paste0}\NormalTok{(}\StringTok{"The }\SpecialCharTok{\textbackslash{}\textquotesingle{}}\StringTok{oil}\SpecialCharTok{\textbackslash{}\textquotesingle{}}\StringTok{ data frame has "}\NormalTok{, }\FunctionTok{nrow}\NormalTok{(oil), }\StringTok{" observations and  "}\NormalTok{, }\FunctionTok{ncol}\NormalTok{(oil), }\StringTok{" variables. "}\NormalTok{)}
\end{Highlighting}
\end{Shaded}

\begin{verbatim}
## [1] "The 'oil' data frame has 1000 observations and  25 variables. "
\end{verbatim}

\begin{center}\rule{0.5\linewidth}{0.5pt}\end{center}

Now we want to know what are the variables' names in the \texttt{oil}
data frame. They correspond to the descriptions in the
\texttt{Dataset\ Format} section at the description of this project.

\begin{Shaded}
\begin{Highlighting}[]
\CommentTok{\# columns names (oil)}
\NormalTok{oil\_col\_names }\OtherTok{\textless{}{-}} \FunctionTok{colnames}\NormalTok{(oil)}
\FunctionTok{cat}\NormalTok{(}\StringTok{\textquotesingle{}Variables in the }\SpecialCharTok{\textbackslash{}\textquotesingle{}}\StringTok{oil}\SpecialCharTok{\textbackslash{}\textquotesingle{}}\StringTok{ data frame: }\SpecialCharTok{\textbackslash{}n\textbackslash{}n}\StringTok{\textquotesingle{}}\NormalTok{)}
\end{Highlighting}
\end{Shaded}

\begin{verbatim}
## Variables in the 'oil' data frame:
\end{verbatim}

\begin{Shaded}
\begin{Highlighting}[]
\ControlFlowTok{for}\NormalTok{ (i }\ControlFlowTok{in}\NormalTok{ (}\DecValTok{1}\SpecialCharTok{:}\FunctionTok{ncol}\NormalTok{(oil))) \{}\FunctionTok{print}\NormalTok{(}\FunctionTok{paste0}\NormalTok{(}\StringTok{"Column number: "}\NormalTok{,i, }\StringTok{". Variable name: "}\NormalTok{, oil\_col\_names[i]))\}}
\end{Highlighting}
\end{Shaded}

\begin{verbatim}
## [1] "Column number: 1. Variable name: OILPRICE"
## [1] "Column number: 2. Variable name: CL2_log"
## [1] "Column number: 3. Variable name: CL3_log"
## [1] "Column number: 4. Variable name: CL4_log"
## [1] "Column number: 5. Variable name: CL5_log"
## [1] "Column number: 6. Variable name: CL6_log"
## [1] "Column number: 7. Variable name: CL7_log"
## [1] "Column number: 8. Variable name: CL8_log"
## [1] "Column number: 9. Variable name: CL9_log"
## [1] "Column number: 10. Variable name: CL10_log"
## [1] "Column number: 11. Variable name: CL11_log"
## [1] "Column number: 12. Variable name: CL12_log"
## [1] "Column number: 13. Variable name: CL13_log"
## [1] "Column number: 14. Variable name: CL14_log"
## [1] "Column number: 15. Variable name: CL15_log"
## [1] "Column number: 16. Variable name: BDIY_log"
## [1] "Column number: 17. Variable name: SPX_log"
## [1] "Column number: 18. Variable name: DX1_log"
## [1] "Column number: 19. Variable name: GC1_log"
## [1] "Column number: 20. Variable name: HO1_log"
## [1] "Column number: 21. Variable name: USCI_log"
## [1] "Column number: 22. Variable name: GNR_log"
## [1] "Column number: 23. Variable name: SHCOMP_log"
## [1] "Column number: 24. Variable name: FTSE_log"
## [1] "Column number: 25. Variable name: respLAG"
\end{verbatim}

\begin{Shaded}
\begin{Highlighting}[]
\CommentTok{\# paste0("Variable name: ", col\_names)}
\end{Highlighting}
\end{Shaded}

\begin{longtable}[]{@{}
  >{\raggedright\arraybackslash}p{(\columnwidth - 0\tabcolsep) * \real{1.00}}@{}}
\toprule
\endhead
Checking if there are any null values in the \texttt{oil} data frame: \\
\texttt{r\ is.null(oil)} \\
\texttt{\#\#\ {[}1{]}\ FALSE} \\
\texttt{r\ sum(is.na(oil))} \\
\texttt{\#\#\ {[}1{]}\ 0} \\
Great! There is no null value in the \texttt{oil} data frame \\
\bottomrule
\end{longtable}

It is time to create a data frame called \texttt{df} that contains all
\texttt{oil} data frame variables except the log prices of the 2 to 15
months ahead WTI oil contracts traded by NYMEX, the so-called
\texttt{CL\#\_log} variables.

Our new \texttt{df} data frame holds only the commodities and stocks
indexes, the predictor variable \texttt{OILPRICE}, and its lagged
version, the \texttt{respLAG} variable.

The idea is to remove unnecessary columns that may add noise to our
data:

\begin{Shaded}
\begin{Highlighting}[]
\CommentTok{\# Creating a new data frame called \textquotesingle{}df\textquotesingle{} without the past contracts information.}

\CommentTok{\# According to the number of columns and their names in \textquotesingle{}oil\_col\_names\textquotesingle{}, lets}
\CommentTok{\# remove the columns 2 to 15 representing the \textquotesingle{}CL\#\_log\textquotesingle{}s variables}
\NormalTok{df }\OtherTok{=}\NormalTok{ oil[,}\SpecialCharTok{{-}}\DecValTok{2}\SpecialCharTok{:{-}}\DecValTok{15}\NormalTok{]}
\FunctionTok{head}\NormalTok{(df)}
\end{Highlighting}
\end{Shaded}

\begin{verbatim}
##   OILPRICE BDIY_log  SPX_log  DX1_log  GC1_log  HO1_log USCI_log  GNR_log
## 1 4.640923 6.850126 7.221624 4.386554 7.413367 1.136197 4.108412 3.917806
## 2 4.633077 6.850126 7.235309 4.379762 7.419680 1.152564 4.120986 3.942552
## 3 4.634049 6.879356 7.222756 4.387449 7.418481 1.155182 4.115127 3.923952
## 4 4.646312 6.882437 7.222252 4.383675 7.410347 1.136743 4.103965 3.925531
## 5 4.631520 6.896694 7.237620 4.382364 7.399704 1.139946 4.107096 3.941970
## 6 4.627616 6.913737 7.233556 4.382951 7.404888 1.137256 4.097672 3.936325
##   SHCOMP_log FTSE_log  respLAG
## 1   7.744539 8.636699 4.631812
## 2   7.762536 8.650062 4.640923
## 3   7.766061 8.639729 4.633077
## 4   7.765158 8.642292 4.634049
## 5   7.755763 8.659907 4.646312
## 6   7.775213 8.656137 4.631520
\end{verbatim}

\begin{center}\rule{0.5\linewidth}{0.5pt}\end{center}

Getting the number of rows (observations) and columns (variables) there
are in the \texttt{df} data frame:

\begin{Shaded}
\begin{Highlighting}[]
\FunctionTok{paste0}\NormalTok{(}\StringTok{"The \textquotesingle{}df\textquotesingle{} data frame has "}\NormalTok{, }\FunctionTok{nrow}\NormalTok{(df), }\StringTok{" observations and  "}\NormalTok{, }\FunctionTok{ncol}\NormalTok{(df), }\StringTok{" variables. "}\NormalTok{)}
\end{Highlighting}
\end{Shaded}

\begin{verbatim}
## [1] "The 'df' data frame has 1000 observations and  11 variables. "
\end{verbatim}

`df' data frame column names:

\begin{Shaded}
\begin{Highlighting}[]
\CommentTok{\# columns names (df)}
\NormalTok{df\_col\_names }\OtherTok{\textless{}{-}} \FunctionTok{colnames}\NormalTok{(df)}
\FunctionTok{cat}\NormalTok{(}\StringTok{\textquotesingle{}Variables in the }\SpecialCharTok{\textbackslash{}\textquotesingle{}}\StringTok{df}\SpecialCharTok{\textbackslash{}\textquotesingle{}}\StringTok{ data frame: }\SpecialCharTok{\textbackslash{}n\textbackslash{}n}\StringTok{\textquotesingle{}}\NormalTok{)}
\end{Highlighting}
\end{Shaded}

\begin{verbatim}
## Variables in the 'df' data frame:
\end{verbatim}

\begin{Shaded}
\begin{Highlighting}[]
\ControlFlowTok{for}\NormalTok{ (i }\ControlFlowTok{in}\NormalTok{ (}\DecValTok{1}\SpecialCharTok{:}\FunctionTok{ncol}\NormalTok{(df))) \{}\FunctionTok{print}\NormalTok{(}\FunctionTok{paste0}\NormalTok{(}\StringTok{"Column number: "}\NormalTok{,i, }\StringTok{". Variable name: "}\NormalTok{, df\_col\_names[i]))\}}
\end{Highlighting}
\end{Shaded}

\begin{verbatim}
## [1] "Column number: 1. Variable name: OILPRICE"
## [1] "Column number: 2. Variable name: BDIY_log"
## [1] "Column number: 3. Variable name: SPX_log"
## [1] "Column number: 4. Variable name: DX1_log"
## [1] "Column number: 5. Variable name: GC1_log"
## [1] "Column number: 6. Variable name: HO1_log"
## [1] "Column number: 7. Variable name: USCI_log"
## [1] "Column number: 8. Variable name: GNR_log"
## [1] "Column number: 9. Variable name: SHCOMP_log"
## [1] "Column number: 10. Variable name: FTSE_log"
## [1] "Column number: 11. Variable name: respLAG"
\end{verbatim}

\begin{Shaded}
\begin{Highlighting}[]
\CommentTok{\# paste0("Variable name: ", col\_names)}
\end{Highlighting}
\end{Shaded}

Printing a statistic summary to overview the \texttt{df} features
values:

\begin{Shaded}
\begin{Highlighting}[]
\CommentTok{\# Let us see some basic statistics of our new \textquotesingle{}df\textquotesingle{} data frame}
\FunctionTok{summary}\NormalTok{(df)}
\end{Highlighting}
\end{Shaded}

\begin{verbatim}
##     OILPRICE        BDIY_log        SPX_log         DX1_log     
##  Min.   :3.266   Min.   :5.670   Min.   :7.153   Min.   :4.369  
##  1st Qu.:3.966   1st Qu.:6.596   1st Qu.:7.354   1st Qu.:4.391  
##  Median :4.517   Median :6.806   Median :7.531   Median :4.417  
##  Mean   :4.309   Mean   :6.787   Mean   :7.481   Mean   :4.459  
##  3rd Qu.:4.580   3rd Qu.:7.011   3rd Qu.:7.611   3rd Qu.:4.557  
##  Max.   :4.705   Max.   :7.757   Max.   :7.664   Max.   :4.613  
##     GC1_log         HO1_log           USCI_log        GNR_log     
##  Min.   :6.956   Min.   :-0.1442   Min.   :3.650   Min.   :3.317  
##  1st Qu.:7.089   1st Qu.: 0.6220   1st Qu.:3.838   1st Qu.:3.787  
##  Median :7.159   Median : 1.0547   Median :4.021   Median :3.868  
##  Mean   :7.192   Mean   : 0.8600   Mean   :3.962   Mean   :3.818  
##  3rd Qu.:7.345   3rd Qu.: 1.1013   3rd Qu.:4.070   3rd Qu.:3.920  
##  Max.   :7.491   Max.   : 1.1877   Max.   :4.148   Max.   :3.985  
##    SHCOMP_log       FTSE_log        respLAG     
##  Min.   :7.576   Min.   :8.568   Min.   :3.266  
##  1st Qu.:7.652   1st Qu.:8.716   1st Qu.:3.966  
##  Median :7.734   Median :8.778   Median :4.517  
##  Mean   :7.840   Mean   :8.760   Mean   :4.310  
##  3rd Qu.:8.032   3rd Qu.:8.813   3rd Qu.:4.580  
##  Max.   :8.550   Max.   :8.868   Max.   :4.705
\end{verbatim}

It seems we don't have any outliers or any missing inputs values in
\texttt{df}. Let's continue our exploratory analysis.

\hypertarget{plotting-oilprice-relation-with-other-variables}{%
\paragraph{\texorpdfstring{Plotting \texttt{OILPRICE} relation with
other
variables}{Plotting OILPRICE relation with other variables}}\label{plotting-oilprice-relation-with-other-variables}}

To create a strategy that can solve our problem, we need to understand
the relationship between Y and X's, the relation between the predictor
variable and the response variables. For that will can use a scatter
plot to make the relation graphically explicit:

\begin{Shaded}
\begin{Highlighting}[]
\NormalTok{p1 }\OtherTok{\textless{}{-}} \FunctionTok{ggplot}\NormalTok{(df, }\FunctionTok{aes}\NormalTok{(SPX\_log, OILPRICE)) }\SpecialCharTok{+} \FunctionTok{geom\_point}\NormalTok{()}
\NormalTok{p2 }\OtherTok{\textless{}{-}} \FunctionTok{ggplot}\NormalTok{(df, }\FunctionTok{aes}\NormalTok{(DX1\_log, OILPRICE)) }\SpecialCharTok{+}  \FunctionTok{geom\_point}\NormalTok{()}
\NormalTok{p3 }\OtherTok{\textless{}{-}} \FunctionTok{ggplot}\NormalTok{(df, }\FunctionTok{aes}\NormalTok{(BDIY\_log, OILPRICE)) }\SpecialCharTok{+}  \FunctionTok{geom\_point}\NormalTok{()}
\NormalTok{p4 }\OtherTok{\textless{}{-}} \FunctionTok{ggplot}\NormalTok{(df, }\FunctionTok{aes}\NormalTok{(GC1\_log, OILPRICE)) }\SpecialCharTok{+}  \FunctionTok{geom\_point}\NormalTok{()}
\NormalTok{p5 }\OtherTok{\textless{}{-}} \FunctionTok{ggplot}\NormalTok{(df, }\FunctionTok{aes}\NormalTok{(HO1\_log, OILPRICE)) }\SpecialCharTok{+}  \FunctionTok{geom\_point}\NormalTok{()}
\NormalTok{p6 }\OtherTok{\textless{}{-}} \FunctionTok{ggplot}\NormalTok{(df, }\FunctionTok{aes}\NormalTok{(FTSE\_log, OILPRICE)) }\SpecialCharTok{+}  \FunctionTok{geom\_point}\NormalTok{()}
\NormalTok{p7 }\OtherTok{\textless{}{-}} \FunctionTok{ggplot}\NormalTok{(df, }\FunctionTok{aes}\NormalTok{(USCI\_log, OILPRICE)) }\SpecialCharTok{+}  \FunctionTok{geom\_point}\NormalTok{()}
\NormalTok{p8 }\OtherTok{\textless{}{-}} \FunctionTok{ggplot}\NormalTok{(df, }\FunctionTok{aes}\NormalTok{(GNR\_log, OILPRICE)) }\SpecialCharTok{+}  \FunctionTok{geom\_point}\NormalTok{()}
\NormalTok{p9 }\OtherTok{\textless{}{-}} \FunctionTok{ggplot}\NormalTok{(df, }\FunctionTok{aes}\NormalTok{(SHCOMP\_log, OILPRICE)) }\SpecialCharTok{+}  \FunctionTok{geom\_point}\NormalTok{()}

\NormalTok{plot\_list }\OtherTok{=} \FunctionTok{list}\NormalTok{(p1, p2, p3, p4, p5, p6, p7, p8, p9)}

\FunctionTok{plot\_grid}\NormalTok{(}\AttributeTok{plotlist =}\NormalTok{ plot\_list,}
          \AttributeTok{labels =} \FunctionTok{c}\NormalTok{(}\StringTok{\textquotesingle{}A\textquotesingle{}}\NormalTok{, }\StringTok{\textquotesingle{}B\textquotesingle{}}\NormalTok{, }\StringTok{\textquotesingle{}C\textquotesingle{}}\NormalTok{, }\StringTok{\textquotesingle{}D\textquotesingle{}}\NormalTok{, }\StringTok{\textquotesingle{}E\textquotesingle{}}\NormalTok{, }\StringTok{\textquotesingle{}F\textquotesingle{}}\NormalTok{, }\StringTok{\textquotesingle{}G\textquotesingle{}}\NormalTok{, }\StringTok{\textquotesingle{}H\textquotesingle{}}\NormalTok{, }\StringTok{\textquotesingle{}I\textquotesingle{}}\NormalTok{),}
          \AttributeTok{label\_x =} \FloatTok{0.01}\NormalTok{,}
          \AttributeTok{label\_y =} \FloatTok{0.87}\NormalTok{,}
          \AttributeTok{hjust =} \SpecialCharTok{{-}}\FloatTok{0.5}\NormalTok{,}
          \AttributeTok{vjust =} \SpecialCharTok{{-}}\FloatTok{0.5}\NormalTok{,}
          \AttributeTok{label\_fontfamily =} \StringTok{"serif"}\NormalTok{,}
          \AttributeTok{label\_fontface =} \StringTok{"plain"}\NormalTok{,}
          \AttributeTok{label\_colour =} \StringTok{"blue"}\NormalTok{)}
\end{Highlighting}
\end{Shaded}

\includegraphics{Oil-Project---Ivan-Berlim-Gonçalves_files/figure-latex/unnamed-chunk-11-1.pdf}

The scatter plot above shows different relations between the dependent
(Y) and independent (X's) variables:

\begin{longtable}[]{@{}lll@{}}
\toprule
Graph Label & Variable & Correlation \\
\midrule
\endhead
A & SPX\_log & Unknown Correlation \\
B & DX1\_log & Negative Linear Correlation \\
B & BDIY\_log & Positive Linear Correlation \\
D & GC1\_log & Unknown Correlation \\
E & HO1\_log & Positive Linear Correlation \\
F & FTSE\_log & Unknown Correlation \\
G & USCI\_log & Positive Linear Correlation \\
H & GNR\_log & Positive Linear Correlation \\
I & SHCOMP\_log & Unknown Correlation \\
\bottomrule
\end{longtable}

In general, we expect to see a linear response from the variables
`DX1\_log', `HO1\_log', `USCI\_log' and `GNR\_log' and maybe a
non-linear correlation from `SPX\_log', `GC1\_log', `FTSE\_log' and
`SHCOMP\_log'.

We'll have to find more about the variables to select the best
treatment.

\begin{center}\rule{0.5\linewidth}{0.5pt}\end{center}

\hypertarget{variables-distribution}{%
\paragraph{Variables Distribution}\label{variables-distribution}}

To select the best suitable correlation test and to better interpret
what the results mean, we have to know the variables' values
distribution.

For this, we will plot a series of histograms to visualize their
distribution and finally be able to select a proper correlation method
to ensure good results when modeling.

\begin{Shaded}
\begin{Highlighting}[]
\CommentTok{\# Making the objects in the \textquotesingle{}df\textquotesingle{} data frame searchable}
\FunctionTok{attach}\NormalTok{(df)}

\CommentTok{\# Selecting plot\textquotesingle{}s parameters}
\NormalTok{color1 }\OtherTok{\textless{}{-}} \StringTok{\textquotesingle{}black\textquotesingle{}}
\NormalTok{fill1 }\OtherTok{\textless{}{-}} \StringTok{\textquotesingle{}white\textquotesingle{}}
\NormalTok{alp }\OtherTok{\textless{}{-}}\NormalTok{ .}\DecValTok{2}
\NormalTok{fill2 }\OtherTok{\textless{}{-}} \StringTok{"\#FF6666"}
\NormalTok{x\_inter }\OtherTok{=} \FunctionTok{mean}\NormalTok{(OILPRICE) }
\NormalTok{color2 }\OtherTok{\textless{}{-}} \StringTok{\textquotesingle{}blue\textquotesingle{}}
\NormalTok{lt }\OtherTok{\textless{}{-}} \StringTok{\textquotesingle{}dashed\textquotesingle{}}
\NormalTok{sz }\OtherTok{\textless{}{-}} \DecValTok{1}

\CommentTok{\# Plotting}
\FunctionTok{par}\NormalTok{(}\AttributeTok{mfrow=}\FunctionTok{c}\NormalTok{(}\DecValTok{3}\NormalTok{,}\DecValTok{4}\NormalTok{))}

\NormalTok{h1 }\OtherTok{\textless{}{-}} \FunctionTok{ggplot}\NormalTok{(df, }\FunctionTok{aes}\NormalTok{(}\AttributeTok{x=}\NormalTok{OILPRICE))}\SpecialCharTok{+} \FunctionTok{ggtitle}\NormalTok{(}\StringTok{"Oil Price"}\NormalTok{)}\SpecialCharTok{+}
  \FunctionTok{geom\_histogram}\NormalTok{(}\FunctionTok{aes}\NormalTok{(}\AttributeTok{y=}\NormalTok{..density..), }\AttributeTok{colour=}\NormalTok{color1, }\AttributeTok{fill=}\NormalTok{fill1)}\SpecialCharTok{+}
  \FunctionTok{geom\_density}\NormalTok{(}\AttributeTok{alpha=}\NormalTok{alp, }\AttributeTok{fill=}\NormalTok{fill2)}\SpecialCharTok{+}
  \FunctionTok{geom\_vline}\NormalTok{(}\FunctionTok{aes}\NormalTok{(}\AttributeTok{xintercept=}\FunctionTok{mean}\NormalTok{(OILPRICE)), }\AttributeTok{color=}\NormalTok{color2, }\AttributeTok{linetype=}\NormalTok{lt, }\AttributeTok{size=}\NormalTok{sz)}
\NormalTok{h2 }\OtherTok{\textless{}{-}} \FunctionTok{ggplot}\NormalTok{(df, }\FunctionTok{aes}\NormalTok{(}\AttributeTok{x=}\NormalTok{BDIY\_log))}\SpecialCharTok{+} \FunctionTok{ggtitle}\NormalTok{(}\StringTok{"Baltic Dry"}\NormalTok{)}\SpecialCharTok{+}
  \FunctionTok{geom\_histogram}\NormalTok{(}\FunctionTok{aes}\NormalTok{(}\AttributeTok{y=}\NormalTok{..density..), }\AttributeTok{colour=}\NormalTok{color1, }\AttributeTok{fill=}\NormalTok{fill1)}\SpecialCharTok{+}
  \FunctionTok{geom\_density}\NormalTok{(}\AttributeTok{alpha=}\NormalTok{alp, }\AttributeTok{fill=}\NormalTok{fill2)}\SpecialCharTok{+}
  \FunctionTok{geom\_vline}\NormalTok{(}\FunctionTok{aes}\NormalTok{(}\AttributeTok{xintercept=}\FunctionTok{mean}\NormalTok{(BDIY\_log)), }\AttributeTok{color=}\NormalTok{color2, }\AttributeTok{linetype=}\NormalTok{lt, }\AttributeTok{size=}\NormalTok{sz)}
\NormalTok{h3 }\OtherTok{\textless{}{-}} \FunctionTok{ggplot}\NormalTok{(df, }\FunctionTok{aes}\NormalTok{(}\AttributeTok{x=}\NormalTok{SPX\_log))}\SpecialCharTok{+} \FunctionTok{ggtitle}\NormalTok{(}\StringTok{"S\&P 500 index"}\NormalTok{)}\SpecialCharTok{+}
  \FunctionTok{geom\_histogram}\NormalTok{(}\FunctionTok{aes}\NormalTok{(}\AttributeTok{y=}\NormalTok{..density..), }\AttributeTok{colour=}\NormalTok{color1, }\AttributeTok{fill=}\NormalTok{fill1)}\SpecialCharTok{+}
  \FunctionTok{geom\_density}\NormalTok{(}\AttributeTok{alpha=}\NormalTok{alp, }\AttributeTok{fill=}\NormalTok{fill2)}\SpecialCharTok{+}
  \FunctionTok{geom\_vline}\NormalTok{(}\FunctionTok{aes}\NormalTok{(}\AttributeTok{xintercept=}\FunctionTok{mean}\NormalTok{(SPX\_log)), }\AttributeTok{color=}\NormalTok{color2, }\AttributeTok{linetype=}\NormalTok{lt, }\AttributeTok{size=}\NormalTok{sz)}
\NormalTok{h4 }\OtherTok{\textless{}{-}} \FunctionTok{ggplot}\NormalTok{(df, }\FunctionTok{aes}\NormalTok{(}\AttributeTok{x=}\NormalTok{DX1\_log))}\SpecialCharTok{+} \FunctionTok{ggtitle}\NormalTok{(}\StringTok{"US Dollar Index"}\NormalTok{)}\SpecialCharTok{+}
  \FunctionTok{geom\_histogram}\NormalTok{(}\FunctionTok{aes}\NormalTok{(}\AttributeTok{y=}\NormalTok{..density..), }\AttributeTok{colour=}\NormalTok{color1, }\AttributeTok{fill=}\NormalTok{fill1)}\SpecialCharTok{+}
  \FunctionTok{geom\_density}\NormalTok{(}\AttributeTok{alpha=}\NormalTok{alp, }\AttributeTok{fill=}\NormalTok{fill2)}\SpecialCharTok{+}
  \FunctionTok{geom\_vline}\NormalTok{(}\FunctionTok{aes}\NormalTok{(}\AttributeTok{xintercept=}\FunctionTok{mean}\NormalTok{(DX1\_log)), }\AttributeTok{color=}\NormalTok{color2, }\AttributeTok{linetype=}\NormalTok{lt, }\AttributeTok{size=}\NormalTok{sz)}
\NormalTok{h5 }\OtherTok{\textless{}{-}} \FunctionTok{ggplot}\NormalTok{(df, }\FunctionTok{aes}\NormalTok{(}\AttributeTok{x=}\NormalTok{GC1\_log))}\SpecialCharTok{+} \FunctionTok{ggtitle}\NormalTok{(}\StringTok{"Gold price contract trades"}\NormalTok{)}\SpecialCharTok{+}
  \FunctionTok{geom\_histogram}\NormalTok{(}\FunctionTok{aes}\NormalTok{(}\AttributeTok{y=}\NormalTok{..density..), }\AttributeTok{colour=}\NormalTok{color1, }\AttributeTok{fill=}\NormalTok{fill1)}\SpecialCharTok{+}
  \FunctionTok{geom\_density}\NormalTok{(}\AttributeTok{alpha=}\NormalTok{alp, }\AttributeTok{fill=}\NormalTok{fill2)}\SpecialCharTok{+}
  \FunctionTok{geom\_vline}\NormalTok{(}\FunctionTok{aes}\NormalTok{(}\AttributeTok{xintercept=}\FunctionTok{mean}\NormalTok{(GC1\_log)), }\AttributeTok{color=}\NormalTok{color2, }\AttributeTok{linetype=}\NormalTok{lt, }\AttributeTok{size=}\NormalTok{sz)}
\NormalTok{h6 }\OtherTok{\textless{}{-}} \FunctionTok{ggplot}\NormalTok{(df, }\FunctionTok{aes}\NormalTok{(}\AttributeTok{x=}\NormalTok{HO1\_log))}\SpecialCharTok{+} \FunctionTok{ggtitle}\NormalTok{(}\StringTok{"Heating Oil price contract traded"}\NormalTok{)}\SpecialCharTok{+}
  \FunctionTok{geom\_histogram}\NormalTok{(}\FunctionTok{aes}\NormalTok{(}\AttributeTok{y=}\NormalTok{..density..), }\AttributeTok{colour=}\NormalTok{color1, }\AttributeTok{fill=}\NormalTok{fill1)}\SpecialCharTok{+}
  \FunctionTok{geom\_density}\NormalTok{(}\AttributeTok{alpha=}\NormalTok{alp, }\AttributeTok{fill=}\NormalTok{fill2)}\SpecialCharTok{+}
  \FunctionTok{geom\_vline}\NormalTok{(}\FunctionTok{aes}\NormalTok{(}\AttributeTok{xintercept=}\FunctionTok{mean}\NormalTok{(HO1\_log)), }\AttributeTok{color=}\NormalTok{color2, }\AttributeTok{linetype=}\NormalTok{lt, }\AttributeTok{size=}\NormalTok{sz)}
\NormalTok{h7 }\OtherTok{\textless{}{-}} \FunctionTok{ggplot}\NormalTok{(df, }\FunctionTok{aes}\NormalTok{(}\AttributeTok{x=}\NormalTok{USCI\_log))}\SpecialCharTok{+} \FunctionTok{ggtitle}\NormalTok{(}\StringTok{"US Commodity Index"}\NormalTok{)}\SpecialCharTok{+}
  \FunctionTok{geom\_histogram}\NormalTok{(}\FunctionTok{aes}\NormalTok{(}\AttributeTok{y=}\NormalTok{..density..), }\AttributeTok{colour=}\NormalTok{color1, }\AttributeTok{fill=}\NormalTok{fill1)}\SpecialCharTok{+}
  \FunctionTok{geom\_density}\NormalTok{(}\AttributeTok{alpha=}\NormalTok{alp, }\AttributeTok{fill=}\NormalTok{fill2)}\SpecialCharTok{+}
  \FunctionTok{geom\_vline}\NormalTok{(}\FunctionTok{aes}\NormalTok{(}\AttributeTok{xintercept=}\FunctionTok{mean}\NormalTok{(USCI\_log)), }\AttributeTok{color=}\NormalTok{color2, }\AttributeTok{linetype=}\NormalTok{lt, }\AttributeTok{size=}\NormalTok{sz)}
\NormalTok{h8 }\OtherTok{\textless{}{-}} \FunctionTok{ggplot}\NormalTok{(df, }\FunctionTok{aes}\NormalTok{(}\AttributeTok{x=}\NormalTok{GNR\_log))}\SpecialCharTok{+} \FunctionTok{ggtitle}\NormalTok{(}\StringTok{"S\&P Global Natural Resources Index"}\NormalTok{)}\SpecialCharTok{+}
  \FunctionTok{geom\_histogram}\NormalTok{(}\FunctionTok{aes}\NormalTok{(}\AttributeTok{y=}\NormalTok{..density..), }\AttributeTok{colour=}\NormalTok{color1, }\AttributeTok{fill=}\NormalTok{fill1)}\SpecialCharTok{+}
  \FunctionTok{geom\_density}\NormalTok{(}\AttributeTok{alpha=}\NormalTok{alp, }\AttributeTok{fill=}\NormalTok{fill2)}\SpecialCharTok{+}
  \FunctionTok{geom\_vline}\NormalTok{(}\FunctionTok{aes}\NormalTok{(}\AttributeTok{xintercept=}\FunctionTok{mean}\NormalTok{(GNR\_log)), }\AttributeTok{color=}\NormalTok{color2, }\AttributeTok{linetype=}\NormalTok{lt, }\AttributeTok{size=}\NormalTok{sz)}
\NormalTok{h9 }\OtherTok{\textless{}{-}} \FunctionTok{ggplot}\NormalTok{(df, }\FunctionTok{aes}\NormalTok{(}\AttributeTok{x=}\NormalTok{SHCOMP\_log))}\SpecialCharTok{+} \FunctionTok{ggtitle}\NormalTok{(}\StringTok{"Shanghai Stock Exchange Composite Index"}\NormalTok{)}\SpecialCharTok{+}
  \FunctionTok{geom\_histogram}\NormalTok{(}\FunctionTok{aes}\NormalTok{(}\AttributeTok{y=}\NormalTok{..density..), }\AttributeTok{colour=}\NormalTok{color1, }\AttributeTok{fill=}\NormalTok{fill1)}\SpecialCharTok{+}
  \FunctionTok{geom\_density}\NormalTok{(}\AttributeTok{alpha=}\NormalTok{alp, }\AttributeTok{fill=}\NormalTok{fill2)}\SpecialCharTok{+}
  \FunctionTok{geom\_vline}\NormalTok{(}\FunctionTok{aes}\NormalTok{(}\AttributeTok{xintercept=}\FunctionTok{mean}\NormalTok{(SHCOMP\_log)), }\AttributeTok{color=}\NormalTok{color2, }\AttributeTok{linetype=}\NormalTok{lt, }\AttributeTok{size=}\NormalTok{sz)}
\NormalTok{h10 }\OtherTok{\textless{}{-}} \FunctionTok{ggplot}\NormalTok{(df, }\FunctionTok{aes}\NormalTok{(}\AttributeTok{x=}\NormalTok{FTSE\_log))}\SpecialCharTok{+} \FunctionTok{ggtitle}\NormalTok{(}\StringTok{"FTSE 100 Index"}\NormalTok{)}\SpecialCharTok{+}
  \FunctionTok{geom\_histogram}\NormalTok{(}\FunctionTok{aes}\NormalTok{(}\AttributeTok{y=}\NormalTok{..density..), }\AttributeTok{colour=}\NormalTok{color1, }\AttributeTok{fill=}\NormalTok{fill1)}\SpecialCharTok{+}
  \FunctionTok{geom\_density}\NormalTok{(}\AttributeTok{alpha=}\NormalTok{alp, }\AttributeTok{fill=}\NormalTok{fill2)}\SpecialCharTok{+}
  \FunctionTok{geom\_vline}\NormalTok{(}\FunctionTok{aes}\NormalTok{(}\AttributeTok{xintercept=}\FunctionTok{mean}\NormalTok{(FTSE\_log)), }\AttributeTok{color=}\NormalTok{color2, }\AttributeTok{linetype=}\NormalTok{lt, }\AttributeTok{size=}\NormalTok{sz)}
\NormalTok{h11 }\OtherTok{\textless{}{-}} \FunctionTok{ggplot}\NormalTok{(df, }\FunctionTok{aes}\NormalTok{(}\AttributeTok{x=}\NormalTok{respLAG))}\SpecialCharTok{+} \FunctionTok{ggtitle}\NormalTok{(}\StringTok{"OILPRICE {-} lagged version"}\NormalTok{)}\SpecialCharTok{+}
  \FunctionTok{geom\_histogram}\NormalTok{(}\FunctionTok{aes}\NormalTok{(}\AttributeTok{y=}\NormalTok{..density..), }\AttributeTok{colour=}\NormalTok{color1, }\AttributeTok{fill=}\NormalTok{fill1)}\SpecialCharTok{+}
  \FunctionTok{geom\_density}\NormalTok{(}\AttributeTok{alpha=}\NormalTok{alp, }\AttributeTok{fill=}\NormalTok{fill2)}\SpecialCharTok{+}
  \FunctionTok{geom\_vline}\NormalTok{(}\FunctionTok{aes}\NormalTok{(}\AttributeTok{xintercept=}\FunctionTok{mean}\NormalTok{(respLAG)), }\AttributeTok{color=}\NormalTok{color2, }\AttributeTok{linetype=}\NormalTok{lt, }\AttributeTok{size=}\NormalTok{sz)}

\NormalTok{hist\_list }\OtherTok{=} \FunctionTok{list}\NormalTok{(h1, h2, h3, h4, h5, h6, h7, h8, h9, h10, h11)}

\FunctionTok{plot\_grid}\NormalTok{(}\AttributeTok{plotlist =}\NormalTok{ hist\_list,}
          \AttributeTok{labels =} \FunctionTok{c}\NormalTok{(}\StringTok{\textquotesingle{}A\textquotesingle{}}\NormalTok{, }\StringTok{\textquotesingle{}B\textquotesingle{}}\NormalTok{, }\StringTok{\textquotesingle{}D\textquotesingle{}}\NormalTok{, }\StringTok{\textquotesingle{}E\textquotesingle{}}\NormalTok{, }\StringTok{\textquotesingle{}F\textquotesingle{}}\NormalTok{, }\StringTok{\textquotesingle{}G\textquotesingle{}}\NormalTok{, }\StringTok{\textquotesingle{}H\textquotesingle{}}\NormalTok{, }\StringTok{\textquotesingle{}I\textquotesingle{}}\NormalTok{, }\StringTok{\textquotesingle{}J\textquotesingle{}}\NormalTok{, }\StringTok{\textquotesingle{}K\textquotesingle{}}\NormalTok{, }\StringTok{\textquotesingle{}L\textquotesingle{}}\NormalTok{, }\StringTok{\textquotesingle{}M\textquotesingle{}}\NormalTok{),}
          \AttributeTok{label\_x =} \FloatTok{0.01}\NormalTok{,}
          \AttributeTok{label\_y =} \FloatTok{0.87}\NormalTok{,}
          \AttributeTok{hjust =} \SpecialCharTok{{-}}\FloatTok{0.5}\NormalTok{,}
          \AttributeTok{vjust =} \SpecialCharTok{{-}}\FloatTok{0.5}\NormalTok{,}
          \AttributeTok{label\_fontfamily =} \StringTok{"serif"}\NormalTok{,}
          \AttributeTok{label\_fontface =} \StringTok{"plain"}\NormalTok{,}
          \AttributeTok{label\_colour =} \StringTok{"blue"}\NormalTok{)}
\end{Highlighting}
\end{Shaded}

\begin{verbatim}
## `stat_bin()` using `bins = 30`. Pick better value with `binwidth`.
## `stat_bin()` using `bins = 30`. Pick better value with `binwidth`.
## `stat_bin()` using `bins = 30`. Pick better value with `binwidth`.
## `stat_bin()` using `bins = 30`. Pick better value with `binwidth`.
## `stat_bin()` using `bins = 30`. Pick better value with `binwidth`.
## `stat_bin()` using `bins = 30`. Pick better value with `binwidth`.
## `stat_bin()` using `bins = 30`. Pick better value with `binwidth`.
## `stat_bin()` using `bins = 30`. Pick better value with `binwidth`.
## `stat_bin()` using `bins = 30`. Pick better value with `binwidth`.
## `stat_bin()` using `bins = 30`. Pick better value with `binwidth`.
## `stat_bin()` using `bins = 30`. Pick better value with `binwidth`.
\end{verbatim}

\includegraphics{Oil-Project---Ivan-Berlim-Gonçalves_files/figure-latex/unnamed-chunk-12-1.pdf}

With the exception of the `BDIY\_log', most of the distributions don't
follow a Gaussian (normal) distribution.

\begin{center}\rule{0.5\linewidth}{0.5pt}\end{center}

\hypertarget{variables-correlation}{%
\paragraph{Variables Correlation}\label{variables-correlation}}

As seen, most of the variables don't follow a normal distribution, so it
would be an assertive strategy to make the assumption that the
variables' parameters are distribution-free. For that, we shall use a
non-parametric approach to better suit the different distributions.
Since we have quantitative non-normal variables on input and the same
type in the output, the \texttt{Spearman\ test} would be a great pick
for comparing these features.

The building of the correlation matrix using the Spearman test can help
us understand the correlations and which features have significant
levels, and which don't.

We'll use the P-value to interpret the results. The null hypothesis is
that there is no relation whatsoever between the variables (correlation
= 0).

P-test interpretation wise: higher p-values mean correlations closer to
zero and low p-values mean correlation different than zero between the
variables and Y.

\begin{Shaded}
\begin{Highlighting}[]
\CommentTok{\# Building a correlation Matrix}
\NormalTok{cor.mtest }\OtherTok{\textless{}{-}} \ControlFlowTok{function}\NormalTok{(mat, ...) \{}
\NormalTok{    mat }\OtherTok{\textless{}{-}} \FunctionTok{as.matrix}\NormalTok{(mat)}
\NormalTok{    n }\OtherTok{\textless{}{-}} \FunctionTok{ncol}\NormalTok{(mat)}
\NormalTok{    p.mat}\OtherTok{\textless{}{-}} \FunctionTok{matrix}\NormalTok{(}\ConstantTok{NA}\NormalTok{, n, n)}
    \FunctionTok{diag}\NormalTok{(p.mat) }\OtherTok{\textless{}{-}} \DecValTok{0}
    \ControlFlowTok{for}\NormalTok{ (i }\ControlFlowTok{in} \DecValTok{1}\SpecialCharTok{:}\NormalTok{(n }\SpecialCharTok{{-}} \DecValTok{1}\NormalTok{)) \{}
        \ControlFlowTok{for}\NormalTok{ (j }\ControlFlowTok{in}\NormalTok{ (i }\SpecialCharTok{+} \DecValTok{1}\NormalTok{)}\SpecialCharTok{:}\NormalTok{n) \{}
\NormalTok{            tmp }\OtherTok{\textless{}{-}} \FunctionTok{cor.test}\NormalTok{(mat[, i], mat[, j], ...)}
\NormalTok{            p.mat[i, j] }\OtherTok{\textless{}{-}}\NormalTok{ p.mat[j, i] }\OtherTok{\textless{}{-}}\NormalTok{ tmp}\SpecialCharTok{$}\NormalTok{p.value}
\NormalTok{        \}}
\NormalTok{    \}}
  \FunctionTok{colnames}\NormalTok{(p.mat) }\OtherTok{\textless{}{-}} \FunctionTok{rownames}\NormalTok{(p.mat) }\OtherTok{\textless{}{-}} \FunctionTok{colnames}\NormalTok{(mat)}
\NormalTok{  p.mat}
\NormalTok{\}}

\CommentTok{\# matrix of the p{-}value of the correlation}
\NormalTok{p.mat }\OtherTok{\textless{}{-}} \FunctionTok{cor.mtest}\NormalTok{(df, }\AttributeTok{method =} \StringTok{\textquotesingle{}spearman\textquotesingle{}}\NormalTok{, }\AttributeTok{exact=}\ConstantTok{FALSE}\NormalTok{)}
\NormalTok{p.mat}
\end{Highlighting}
\end{Shaded}

\begin{verbatim}
##                 OILPRICE     BDIY_log       SPX_log       DX1_log       GC1_log
## OILPRICE    0.000000e+00 9.878893e-94  1.164449e-62 3.393567e-222  4.534907e-90
## BDIY_log    9.878893e-94 0.000000e+00  6.792721e-26  1.025300e-66  3.120209e-12
## SPX_log     1.164449e-62 6.792721e-26  0.000000e+00 4.618178e-122 5.540047e-294
## DX1_log    3.393567e-222 1.025300e-66 4.618178e-122  0.000000e+00 7.575202e-189
## GC1_log     4.534907e-90 3.120209e-12 5.540047e-294 7.575202e-189  0.000000e+00
## HO1_log    1.329004e-258 2.946376e-60 1.383726e-143 2.347075e-294 3.759788e-200
## USCI_log   9.042281e-178 6.261781e-36 1.183454e-141 5.987330e-298 5.283737e-256
## GNR_log    1.652549e-190 5.182126e-32  8.890018e-61 1.727438e-279 1.763336e-151
## SHCOMP_log 4.635703e-193 6.546882e-72 1.049770e-111 6.663341e-202 3.874057e-110
## FTSE_log    3.288032e-08 2.359658e-03 8.597015e-120  1.514743e-03  9.516503e-38
## respLAG     0.000000e+00 1.454298e-94  6.127019e-62 5.106686e-225  3.118096e-90
##                  HO1_log      USCI_log       GNR_log    SHCOMP_log
## OILPRICE   1.329004e-258 9.042281e-178 1.652549e-190 4.635703e-193
## BDIY_log    2.946376e-60  6.261781e-36  5.182126e-32  6.546882e-72
## SPX_log    1.383726e-143 1.183454e-141  8.890018e-61 1.049770e-111
## DX1_log    2.347075e-294 5.987330e-298 1.727438e-279 6.663341e-202
## GC1_log    3.759788e-200 5.283737e-256 1.763336e-151 3.874057e-110
## HO1_log     0.000000e+00 1.512872e-270 3.812661e-208 2.106332e-169
## USCI_log   1.512872e-270  0.000000e+00 1.232274e-298 3.848899e-163
## GNR_log    3.812661e-208 1.232274e-298  0.000000e+00 3.509312e-120
## SHCOMP_log 2.106332e-169 3.848899e-163 3.509312e-120  0.000000e+00
## FTSE_log    2.122789e-03  4.528966e-07  7.927264e-06  2.344613e-03
## respLAG    1.779613e-263 5.687291e-180 2.645174e-194 1.610871e-189
##                 FTSE_log       respLAG
## OILPRICE    3.288032e-08  0.000000e+00
## BDIY_log    2.359658e-03  1.454298e-94
## SPX_log    8.597015e-120  6.127019e-62
## DX1_log     1.514743e-03 5.106686e-225
## GC1_log     9.516503e-38  3.118096e-90
## HO1_log     2.122789e-03 1.779613e-263
## USCI_log    4.528966e-07 5.687291e-180
## GNR_log     7.927264e-06 2.645174e-194
## SHCOMP_log  2.344613e-03 1.610871e-189
## FTSE_log    0.000000e+00  2.247735e-08
## respLAG     2.247735e-08  0.000000e+00
\end{verbatim}

As we can see from the correlation matrix above, almost all of the
features have extremely low values. That means that they have a
correlation different than zero to the Y variable. But when we take a
look at the \texttt{FTSE\_log} feature, we notice some relatively higher
values compared to the other features.

If we select a threshold (0,1\%) to the significance levels (the
probability of rejecting the null hypothesis given that it is true), we
will notice that some of the FTSE 100 Indexes stay above of the
threshold. This can be seen below in the graphical correlation matrix.
The squares marked with an `X' mean that the value is higher than the
threshold. In addition, all the other \texttt{FTSE\_log} correlation
values are kinda low compared to other features.

We can say that all but the FTSE 100 index have a different than zero
correlation with \texttt{OILPRICE} variable.

\begin{Shaded}
\begin{Highlighting}[]
\NormalTok{corr\_mat}\OtherTok{=}\FunctionTok{cor}\NormalTok{(df,}\AttributeTok{method=}\StringTok{"s"}\NormalTok{)}
\NormalTok{col }\OtherTok{\textless{}{-}} \FunctionTok{colorRampPalette}\NormalTok{(}\FunctionTok{c}\NormalTok{(}\StringTok{"\#BB4444"}\NormalTok{, }\StringTok{"\#EE9988"}\NormalTok{, }\StringTok{"\#FFFFFF"}\NormalTok{, }\StringTok{"\#77AADD"}\NormalTok{, }\StringTok{"\#4477AA"}\NormalTok{))}
\FunctionTok{corrplot}\NormalTok{(corr\_mat, }\AttributeTok{method=}\StringTok{"color"}\NormalTok{, }\AttributeTok{col=}\FunctionTok{col}\NormalTok{(}\DecValTok{200}\NormalTok{),  }
         \AttributeTok{type=}\StringTok{"upper"}\NormalTok{, }\AttributeTok{order=}\StringTok{"hclust"}\NormalTok{, }
         \AttributeTok{addCoef.col =} \StringTok{"black"}\NormalTok{,}
         \AttributeTok{tl.col=}\StringTok{"black"}\NormalTok{, }\AttributeTok{tl.srt=}\DecValTok{45}\NormalTok{,}
         \CommentTok{\# Combine with significance}
         \AttributeTok{p.mat =}\NormalTok{ p.mat, }\AttributeTok{sig.level =} \FloatTok{0.001}\NormalTok{, }\AttributeTok{insig =} \StringTok{"pch"}\NormalTok{, }
         \CommentTok{\# hide correlation coefficient on the principal diagonal}
         \AttributeTok{diag=}\ConstantTok{FALSE} 
\NormalTok{         )}
\end{Highlighting}
\end{Shaded}

\includegraphics{Oil-Project---Ivan-Berlim-Gonçalves_files/figure-latex/unnamed-chunk-14-1.pdf}

\begin{center}\rule{0.5\linewidth}{0.5pt}\end{center}

\hypertarget{data-preparation-1}{%
\subsubsection{5 - DATA PREPARATION}\label{data-preparation-1}}

\begin{center}\rule{0.5\linewidth}{0.5pt}\end{center}

\hypertarget{variables-transformation}{%
\paragraph{Variables Transformation}\label{variables-transformation}}

As informed in the dataset description, the data values are already
logged transformations of the original format. This means that we don't
need to do any other treatment to the data and we can model it as it is.
In addition, any other additional transformation would difficult the
model interpretation and unnecessarily increase its complexity.

\begin{center}\rule{0.5\linewidth}{0.5pt}\end{center}

\hypertarget{gamlss-model}{%
\paragraph{GAMLSS Model}\label{gamlss-model}}

The GAMLSS library is well suited for our data because of its ability to
handle asymmetric distributions, including high skewed and/or kurtotic
distributions. The package is also ready to deal with addictive
non-parametric functions.

For this project, we'll use the RS algorithm. It usually works with four
parameters: location (\texttt{mu}), scale (\texttt{sigma}), and two
shape parameters (\texttt{nu} and \texttt{tau}), which are estimated by
a penalizing likelihood function \texttt{k}. The model also accepts
additive terms such as smoothing functions for the continuous
explanatory variable x

\hypertarget{family-used-and-performance-indicators}{%
\paragraph{Family used and performance
indicators}\label{family-used-and-performance-indicators}}

For the model to work properly we need to evaluate which distribution
will be passed to it. This was a process of trial and error between the
long list of different distributions available in the GAMLSS library.
The complete list of the distribution is linked in the Bibliography
section.

A lot of different distributions were tested and the Box-Cox power
exponential (BCPE) presented some of the lowest non-parametric test
values (Global Deviation - GD, Akaike Information Criterion - AIC, and
Schwarz Bayesian Criterion - SBC) among all while well-fitting the
OILPRICE distribution curve.

Functions used with the BCPE family distribution:

\begin{Shaded}
\begin{Highlighting}[]
\CommentTok{\# showing available link functions for BCPE distribution parameters}
\FunctionTok{show.link}\NormalTok{(}\StringTok{\textquotesingle{}BCPE\textquotesingle{}}\NormalTok{)}
\end{Highlighting}
\end{Shaded}

\begin{verbatim}
## $mu
## c("inverse", "log", "identity", "own")
## 
## $sigma
## c("inverse", "log", "identity", "own")
## 
## $nu
## c("inverse", "log", "identity", "own")
## 
## $tau
## c("logshiftto1", "log", "identity", "own")
\end{verbatim}

In the graph below, the red line represents the Box-Cox Power
Exponential (BCPE) compared with the \texttt{OILPRICE} distribution
(green line). They show some resemblance and seem sufficiently good for
modeling.

\begin{Shaded}
\begin{Highlighting}[]
\FunctionTok{histDist}\NormalTok{(df}\SpecialCharTok{$}\NormalTok{OILPRICE,}
         \AttributeTok{family =}\NormalTok{ BCPE,}
         \AttributeTok{density =} \ConstantTok{TRUE}\NormalTok{,}
         \AttributeTok{main =} \StringTok{\textquotesingle{}The Oil Price and the fitted BCPE distribution\textquotesingle{}}\NormalTok{,}
         \AttributeTok{ylab =} \StringTok{\textquotesingle{}Frequency\textquotesingle{}}\NormalTok{,}
         \AttributeTok{xlab =} \StringTok{\textquotesingle{}Oil Price\textquotesingle{}}\NormalTok{)}
\end{Highlighting}
\end{Shaded}

\includegraphics{Oil-Project---Ivan-Berlim-Gonçalves_files/figure-latex/unnamed-chunk-16-1.pdf}

\begin{verbatim}
## 
## Family:  c("BCPE", "Box-Cox Power Exponential") 
## Fitting method: "nlminb" 
## 
## Call:  gamlssML(formula = df$OILPRICE, family = "BCPE") 
## 
## Mu Coefficients:
## [1]  4.374
## Sigma Coefficients:
## [1]  -2.877
## Nu Coefficients:
## [1]  9.382
## Tau Coefficients:
## [1]  3.339
## 
##  Degrees of Freedom for the fit: 4 Residual Deg. of Freedom   996 
## Global Deviance:     65.6076 
##             AIC:     73.6076 
##             SBC:     93.2387
\end{verbatim}

\begin{center}\rule{0.5\linewidth}{0.5pt}\end{center}

\hypertarget{model-training}{%
\subsubsection{6 - MODEL TRAINING}\label{model-training}}

While the construction of the project I came to know that the past
traded contracts ends up inserting some amount of noise into the data,
negatively influencing the final model's performance. To avoid this
issue, to keep the simplicity of the project, and to achieve better
computational efficiency aiming to solve the proposed problem we'll have
only one model based on the \texttt{df} data frame. I will not use any
additive terms either, the models without them had slightly better
results. Also, keeping only one model allows us to add more algorithm
training cycles to achieve better results while maintaining a relatively
low script execution time.

\begin{itemize}
\item
  \texttt{model\_df}: used the \texttt{df} data frame and removed the
  past traded contracts and leaving only the lagged variable with no
  additive terms applied.
\item
  Train/test split: 70\% train, 30\% test
\end{itemize}

\begin{Shaded}
\begin{Highlighting}[]
\CommentTok{\# Dividing the \textquotesingle{}df\textquotesingle{} data frame into train and test}
\NormalTok{dt\_df }\OtherTok{=} \FunctionTok{sample}\NormalTok{(}\FunctionTok{nrow}\NormalTok{(df), }\FunctionTok{nrow}\NormalTok{(df)}\SpecialCharTok{*}\NormalTok{.}\DecValTok{7}\NormalTok{, }\AttributeTok{replace =} \ConstantTok{FALSE}\NormalTok{)}
\NormalTok{train\_df }\OtherTok{\textless{}{-}}\NormalTok{ df[dt\_df, ]}
\NormalTok{test\_df }\OtherTok{\textless{}{-}}\NormalTok{ df[}\SpecialCharTok{{-}}\NormalTok{dt\_df, ]}

\FunctionTok{print}\NormalTok{(}\StringTok{\textquotesingle{}Data splitted: 70\% train, 30\% test\textquotesingle{}}\NormalTok{)}
\end{Highlighting}
\end{Shaded}

\begin{verbatim}
## [1] "Data splitted: 70% train, 30% test"
\end{verbatim}

Definition of the model using BCPE as the family of distribution and 350
algorithm cycles. In my case, the algorithm converged at approximately
240 cycles.

I built a conditional check to load the trained model if it exists or to
refit if it doesn't.

\begin{Shaded}
\begin{Highlighting}[]
\DocumentationTok{\#\# check if model exists? If not, refit:}
\ControlFlowTok{if}\NormalTok{(}\FunctionTok{file.exists}\NormalTok{(}\StringTok{"gamlss\_oil\_model.rda"}\NormalTok{)) \{}
    \DocumentationTok{\#\# load model}
    \FunctionTok{load}\NormalTok{(}\StringTok{"gamlss\_oil\_model.rda"}\NormalTok{)}
\NormalTok{\} }\ControlFlowTok{else}\NormalTok{ \{}
    \DocumentationTok{\#\# (re)fit the model}
\NormalTok{    model\_df }\OtherTok{\textless{}{-}} \FunctionTok{gamlss}\NormalTok{(}\AttributeTok{formula =}\NormalTok{ OILPRICE }\SpecialCharTok{\textasciitilde{}}\NormalTok{ ., }\AttributeTok{family =}\NormalTok{ BCPE, }\AttributeTok{data =}\NormalTok{ train\_df, }\AttributeTok{control =} \FunctionTok{gamlss.control}\NormalTok{(}\AttributeTok{n.cyc =} \DecValTok{350}\NormalTok{)) }
\NormalTok{\}}

\CommentTok{\# save the model}
\FunctionTok{save}\NormalTok{(model\_df, }\AttributeTok{file =} \StringTok{"gamlss\_oil\_model.rda"}\NormalTok{)}
\end{Highlighting}
\end{Shaded}

Printing a statistic summary on the model

\begin{Shaded}
\begin{Highlighting}[]
\FunctionTok{summary}\NormalTok{(model\_df)}
\end{Highlighting}
\end{Shaded}

\begin{verbatim}
## Warning in summary.gamlss(model_df): summary: vcov has failed, option qr is used instead
\end{verbatim}

\begin{verbatim}
## ******************************************************************
## Family:  c("BCPE", "Box-Cox Power Exponential") 
## 
## Call:  gamlss(formula = OILPRICE ~ ., family = BCPE, data = train_df,  
##     control = gamlss.control(n.cyc = 350)) 
## 
## Fitting method: RS() 
## 
## ------------------------------------------------------------------
## Mu link function:  identity
## Mu Coefficients:
##              Estimate Std. Error t value Pr(>|t|)    
## (Intercept)  0.447840   0.222206   2.015 0.044248 *  
## BDIY_log    -0.001767   0.001683  -1.050 0.294105    
## SPX_log     -0.101904   0.012123  -8.406 2.43e-16 ***
## DX1_log     -0.021347   0.022834  -0.935 0.350173    
## GC1_log     -0.068064   0.011230  -6.061 2.23e-09 ***
## HO1_log      0.004727   0.010092   0.468 0.639656    
## USCI_log     0.063255   0.017820   3.550 0.000412 ***
## GNR_log      0.009219   0.017327   0.532 0.594862    
## SHCOMP_log  -0.003182   0.004702  -0.677 0.498855    
## FTSE_log     0.093224   0.017879   5.214 2.44e-07 ***
## respLAG      0.960363   0.007368 130.339  < 2e-16 ***
## ---
## Signif. codes:  0 '***' 0.001 '**' 0.01 '*' 0.05 '.' 0.1 ' ' 1
## 
## ------------------------------------------------------------------
## Sigma link function:  log
## Sigma Coefficients:
##             Estimate Std. Error t value Pr(>|t|)    
## (Intercept) -5.14670    0.04404  -116.9   <2e-16 ***
## ---
## Signif. codes:  0 '***' 0.001 '**' 0.01 '*' 0.05 '.' 0.1 ' ' 1
## 
## ------------------------------------------------------------------
## Nu link function:  identity 
## Nu Coefficients:
##             Estimate Std. Error t value Pr(>|t|)
## (Intercept)   -4.405      7.251  -0.607    0.544
## 
## ------------------------------------------------------------------
## Tau link function:  log 
## Tau Coefficients:
##             Estimate Std. Error t value Pr(>|t|)    
## (Intercept) -0.30652    0.05447  -5.627 2.66e-08 ***
## ---
## Signif. codes:  0 '***' 0.001 '**' 0.01 '*' 0.05 '.' 0.1 ' ' 1
## 
## ------------------------------------------------------------------
## No. of observations in the fit:  700 
## Degrees of Freedom for the fit:  14
##       Residual Deg. of Freedom:  686 
##                       at cycle:  240 
##  
## Global Deviance:     -3434.44 
##             AIC:     -3406.44 
##             SBC:     -3342.725 
## ******************************************************************
\end{verbatim}

Due to the elevated significance values presented at the correlation
matrix and its respective plot, we have to be aware of signs of
overfitting. Therefore, the models' evaluation metrics AIC and SBC are
not the best metrics to evaluate the model's performance. Global
deviance is a good performance metric, but again, it also can suffer the
effects of possible overfitting.

\begin{center}\rule{0.5\linewidth}{0.5pt}\end{center}

\hypertarget{model-evaluation}{%
\subsubsection{7 - MODEL EVALUATION}\label{model-evaluation}}

Now let's predict the test subset, calculate its percentage deviation
and analyze the results.

\begin{Shaded}
\begin{Highlighting}[]
\CommentTok{\# Prediction}
\NormalTok{test\_df}\SpecialCharTok{$}\NormalTok{pred }\OtherTok{\textless{}{-}} \FunctionTok{predict}\NormalTok{(model\_df, }\AttributeTok{newdata=}\NormalTok{test\_df, }\AttributeTok{type =} \StringTok{"response"}\NormalTok{)}
\end{Highlighting}
\end{Shaded}

\begin{verbatim}
## Warning in predict.gamlss(model_df, newdata = test_df, type = "response"): There is a discrepancy  between the original and the re-fit 
##  used to achieve 'safe' predictions 
## 
\end{verbatim}

\begin{Shaded}
\begin{Highlighting}[]
\CommentTok{\# Deviation calculation}
\NormalTok{test\_df}\SpecialCharTok{$}\NormalTok{pred.deviation  }\OtherTok{\textless{}{-}} \FunctionTok{round}\NormalTok{(test\_df}\SpecialCharTok{$}\NormalTok{pred}\SpecialCharTok{/}\NormalTok{test\_df}\SpecialCharTok{$}\NormalTok{OILPRICE,}\DecValTok{10}\NormalTok{)}
\NormalTok{test\_df}\SpecialCharTok{$}\NormalTok{pred.deviation  }\OtherTok{\textless{}{-}}\NormalTok{ test\_df}\SpecialCharTok{$}\NormalTok{pred.deviation}\DecValTok{{-}1}

\CommentTok{\# Plotting prediction\textquotesingle{}s deviation}
\FunctionTok{par}\NormalTok{(}\AttributeTok{mfrow=}\FunctionTok{c}\NormalTok{(}\DecValTok{1}\NormalTok{,}\DecValTok{2}\NormalTok{))}
\FunctionTok{hist}\NormalTok{(}\DecValTok{100}\SpecialCharTok{*}\NormalTok{test\_df}\SpecialCharTok{$}\NormalTok{pred.deviation, }\AttributeTok{main=}\StringTok{"Prediction +{-}2\% deviation interval"}\NormalTok{, }\AttributeTok{xlab =} \StringTok{"Deviation percentage"}\NormalTok{, }\AttributeTok{labels =}\NormalTok{ T, }\AttributeTok{xlim =}\FunctionTok{c}\NormalTok{(}\SpecialCharTok{{-}}\DecValTok{2}\NormalTok{,}\DecValTok{2}\NormalTok{), }\AttributeTok{ylim =} \FunctionTok{c}\NormalTok{(}\DecValTok{0}\NormalTok{,}\DecValTok{150}\NormalTok{))}
\FunctionTok{hist}\NormalTok{(}\DecValTok{100}\SpecialCharTok{*}\NormalTok{test\_df}\SpecialCharTok{$}\NormalTok{pred.deviation, }\AttributeTok{main=}\StringTok{"Prediction +{-}.5\% deviation interval"}\NormalTok{, }\AttributeTok{xlab =} \StringTok{"Deviation percentage"}\NormalTok{, }\AttributeTok{labels =}\NormalTok{ T, }\AttributeTok{xlim =}\FunctionTok{c}\NormalTok{(}\SpecialCharTok{{-}}\FloatTok{0.5}\NormalTok{,}\FloatTok{0.5}\NormalTok{), }\AttributeTok{ylim =} \FunctionTok{c}\NormalTok{(}\DecValTok{0}\NormalTok{,}\DecValTok{150}\NormalTok{), }\AttributeTok{breaks =} \DecValTok{32}\NormalTok{)}
\end{Highlighting}
\end{Shaded}

\includegraphics{Oil-Project---Ivan-Berlim-Gonçalves_files/figure-latex/unnamed-chunk-20-1.pdf}

The interval of 1\% deviation of the predictions shows us an important
metric to evaluate the model. We can see that the range of deviation in
the first graph is 2\%. Considering that we are dealing with price
values, a 2\% deviation is a very high deviation. Given the volume of
negotiations and the values involved, the expected oscillation in this
business segment is about 1\%. An excellent deviation oscillation range
would be considered about 0.5\%.

Also, this information is a good metric to evaluate the model's
performance. The ideal scenario would be to feed additional information
to the model to see if it is overfitted. However, while elaborating on
this project, I came across different models and different models
metrics and I'm confident that the model is not under fitted or
overfitted.

Finally, let's see the percentage of values between -+1\% and +-0.5\%
deviation range and analyze if the results are sufficient for our
problem.

\begin{Shaded}
\begin{Highlighting}[]
\NormalTok{test\_dev }\OtherTok{\textless{}{-}} \DecValTok{100}\SpecialCharTok{*}\NormalTok{test\_df}\SpecialCharTok{$}\NormalTok{pred.deviation}
\NormalTok{test\_1dev }\OtherTok{\textless{}{-}} \FunctionTok{sum}\NormalTok{(}\FunctionTok{abs}\NormalTok{(test\_dev) }\SpecialCharTok{\textless{}=} \DecValTok{1}\NormalTok{)}
\NormalTok{test\_05dev }\OtherTok{\textless{}{-}} \FunctionTok{sum}\NormalTok{(}\FunctionTok{abs}\NormalTok{(test\_dev) }\SpecialCharTok{\textless{}=} \FloatTok{0.5}\NormalTok{)}
\NormalTok{test\_size }\OtherTok{\textless{}{-}} \FunctionTok{nrow}\NormalTok{(test\_df)}

\NormalTok{test\_in\_1dev }\OtherTok{=}\NormalTok{ (test\_1dev }\SpecialCharTok{/}\NormalTok{ test\_size)}\SpecialCharTok{*}\DecValTok{100}
\NormalTok{test\_in\_05dev }\OtherTok{=}\NormalTok{ (test\_05dev }\SpecialCharTok{/}\NormalTok{ test\_size)}\SpecialCharTok{*}\DecValTok{100}

\FunctionTok{paste0}\NormalTok{(}\FunctionTok{round}\NormalTok{(test\_in\_1dev,}\DecValTok{2}\NormalTok{), }\StringTok{\textquotesingle{}\% of the predicted values are in a +{-}1\% deviation margin to the real values\textquotesingle{}}\NormalTok{)}
\end{Highlighting}
\end{Shaded}

\begin{verbatim}
## [1] "93.33% of the predicted values are in a +-1% deviation margin to the real values"
\end{verbatim}

\begin{Shaded}
\begin{Highlighting}[]
\FunctionTok{paste0}\NormalTok{(}\FunctionTok{round}\NormalTok{(test\_in\_05dev,}\DecValTok{2}\NormalTok{), }\StringTok{\textquotesingle{}\% of the predicted values are in a +{-}0.5\% deviation margin to the real values\textquotesingle{}}\NormalTok{)}
\end{Highlighting}
\end{Shaded}

\begin{verbatim}
## [1] "79% of the predicted values are in a +-0.5% deviation margin to the real values"
\end{verbatim}

We have achieved our goal! 93\% of the predicted values were +-1\% in a
deviation margin, and 79\% in a +-0.5\% margin.

\begin{center}\rule{0.5\linewidth}{0.5pt}\end{center}

\hypertarget{residuals}{%
\paragraph{Residuals}\label{residuals}}

Errors and residuals are two closely related and easily confused
measures of the deviation. The error is the predicted value minus the
true value and the residual is the difference between the observed value
and the estimated value of the quantity of interest

\begin{Shaded}
\begin{Highlighting}[]
\FunctionTok{plot}\NormalTok{(model\_df)}
\end{Highlighting}
\end{Shaded}

\includegraphics{Oil-Project---Ivan-Berlim-Gonçalves_files/figure-latex/unnamed-chunk-22-1.pdf}

\begin{verbatim}
## ******************************************************************
##        Summary of the Quantile Residuals
##                            mean   =  -0.05313394 
##                        variance   =  0.9967813 
##                coef. of skewness  =  0.03337896 
##                coef. of kurtosis  =  2.92972 
## Filliben correlation coefficient  =  0.9995241 
## ******************************************************************
\end{verbatim}

As we can see, the density estimate follows a normal distribution, and
consequently, the observed sample percentiles of the residuals are
linear. This means that the error terms are normally distributed.

\begin{center}\rule{0.5\linewidth}{0.5pt}\end{center}

\hypertarget{worm-plot}{%
\paragraph{Worm Plot}\label{worm-plot}}

The same thing reflects on the worm plot, where all of the residuals are
in or at the border of the dotted confidence bands. This indicates the
adequacy in modeling the distribution.

\begin{Shaded}
\begin{Highlighting}[]
\CommentTok{\# Plotting residuals}
\FunctionTok{wp}\NormalTok{(model\_df)}
\end{Highlighting}
\end{Shaded}

\includegraphics{Oil-Project---Ivan-Berlim-Gonçalves_files/figure-latex/unnamed-chunk-23-1.pdf}

\begin{center}\rule{0.5\linewidth}{0.5pt}\end{center}

\hypertarget{metrics-non-parametric-tests}{%
\paragraph{Metrics: Non-parametric
Tests}\label{metrics-non-parametric-tests}}

Let's see the non-parametric numbers

\begin{Shaded}
\begin{Highlighting}[]
\CommentTok{\# Getting each model\textquotesingle{}s non parametric test}

\NormalTok{GD }\OtherTok{=} \FunctionTok{c}\NormalTok{(model\_df}\SpecialCharTok{$}\NormalTok{G.deviance) }\CommentTok{\#Deviation}
\NormalTok{AIC }\OtherTok{=} \FunctionTok{c}\NormalTok{(model\_df}\SpecialCharTok{$}\NormalTok{aic)}\CommentTok{\# Overfitting}
\NormalTok{SBC }\OtherTok{=} \FunctionTok{c}\NormalTok{(model\_df}\SpecialCharTok{$}\NormalTok{sbc)}\CommentTok{\# Overfitting}

\FunctionTok{paste0}\NormalTok{(}\StringTok{\textquotesingle{}The model Global Deviation {-} GD was: \textquotesingle{}}\NormalTok{, }\FunctionTok{round}\NormalTok{(GD,}\DecValTok{2}\NormalTok{))}
\end{Highlighting}
\end{Shaded}

\begin{verbatim}
## [1] "The model Global Deviation - GD was: -3434.44"
\end{verbatim}

\begin{Shaded}
\begin{Highlighting}[]
\FunctionTok{paste0}\NormalTok{(}\StringTok{\textquotesingle{}The model Akaike Information Criterion {-} AIC was: \textquotesingle{}}\NormalTok{, }\FunctionTok{round}\NormalTok{(AIC,}\DecValTok{2}\NormalTok{))}
\end{Highlighting}
\end{Shaded}

\begin{verbatim}
## [1] "The model Akaike Information Criterion - AIC was: -3406.44"
\end{verbatim}

\begin{Shaded}
\begin{Highlighting}[]
\FunctionTok{paste0}\NormalTok{(}\StringTok{\textquotesingle{}The model Schwarz Bayesian Criterion {-} SBC was: \textquotesingle{}}\NormalTok{, }\FunctionTok{round}\NormalTok{(SBC,}\DecValTok{2}\NormalTok{))}
\end{Highlighting}
\end{Shaded}

\begin{verbatim}
## [1] "The model Schwarz Bayesian Criterion - SBC was: -3342.73"
\end{verbatim}

\begin{center}\rule{0.5\linewidth}{0.5pt}\end{center}

\hypertarget{bonus-final-price-in-us-dollar}{%
\paragraph{BONUS: Final Price in US
Dollar}\label{bonus-final-price-in-us-dollar}}

Finally lets transform the predictions in to real world money. Applying
a exponential function to the predicted values can restore the real
world values of the oil. Probably in US Dollars. Lets take a peak at the
values and save it as a .csv file.

\begin{Shaded}
\begin{Highlighting}[]
\NormalTok{oil\_dollar }\OtherTok{\textless{}{-}} \FunctionTok{data.frame}\NormalTok{(}\FunctionTok{exp}\NormalTok{(test\_df}\SpecialCharTok{$}\NormalTok{pred))}
\FunctionTok{colnames}\NormalTok{(oil\_dollar) }\OtherTok{\textless{}{-}} \FunctionTok{c}\NormalTok{(}\StringTok{"Oil price in US Dollar"}\NormalTok{)}
\FunctionTok{head}\NormalTok{(oil\_dollar)}
\end{Highlighting}
\end{Shaded}

\begin{verbatim}
##   Oil price in US Dollar
## 1              104.39876
## 2              102.81482
## 3              104.88899
## 4              105.17268
## 5               98.65589
## 6               97.28388
\end{verbatim}

\begin{Shaded}
\begin{Highlighting}[]
\FunctionTok{write.csv}\NormalTok{(}\AttributeTok{x=}\NormalTok{oil\_dollar, }\AttributeTok{file=}\StringTok{"oil\_predictions.csv"}\NormalTok{)}
\end{Highlighting}
\end{Shaded}

\begin{center}\rule{0.5\linewidth}{0.5pt}\end{center}

\hypertarget{conclusions}{%
\subsubsection{8 - CONCLUSIONS}\label{conclusions}}

Not using the past contracts variables \texttt{CL\#\_log} was a good
strategy to not impose unnecessary noise into the data. The features
used in \texttt{df} data frame were sufficiently good and at the same
time kept the model simple and effective. The high significance in the
correlation matrix imposed a doubt whether the model had overfitted or
not but the model results on the test data and the robust analysis on
its residuals supports the idea that the model is just on the right
spot. The model returned a high performance on predicting the values,
delivering a high-quality standard prevision, and more important:
answered the problem question and achieved the goals.

As said during the execution of this project, the model could predict
93\% of the values within 1\% or less deviation margin, and 78\%
accuracy on 0.5\% deviation margin. Maybe some tuning on the weights and
control parameters of the model could have improved the results, and
maybe some additional tuning on the previous models that I used with
additive terms such as Cubic Splines could deliver better results. But
as the time was a little bit short and I'm continuously learning, maybe
I'll come back soon to this project to continue improving its results.

\hypertarget{personal-conclusion}{%
\subsubsection{9 - PERSONAL CONCLUSION}\label{personal-conclusion}}

It was a very challenging project and I'm very glad that I could achieve
good results. I tried to bring some of the research methods that I find
relevant to the real job market while keeping it simple so that you, the
reader, could have a relaxed, but still learningful, experience.

I hope you enjoyed it! Any feedback doesn't hesitate to contact me:

LinkedIn profile:
\url{https://www.linkedin.com/in/ivan-berlim-gonçalves/?locale=en_US}

\hypertarget{bibliography}{%
\subsubsection{10 - BIBLIOGRAPHY}\label{bibliography}}

\begin{itemize}
\item
  Stasinopoulos, M., Rigby, R. A., \& Akantziliotou, C. (2008).
  Instructions on how to use the GAMLSS package in R. 206.
  \url{http://www.gamlss.com/wp-content/uploads/2013/01/gamlss-manual.pdf}
\item
  Rigby, R., Stasinopoulos, M., Heller, G., \& De Bastiani, F. (2019).
  Distributions for Modelling Location, Scale and Shape: Using GAMLSS in
  R. CRC Press.
  \url{http://www.gamlss.com/wp-content/uploads/2018/01/DistributionsForModellingLocationScaleandShape.pdf}
\item
  Stasinopolulos, M., Rigby, R. A., Voudouris, V., Heler, G., \&
  Bastiani De, F. (2017). Flexible regression and smoothing the GAMLSS
  packages in R. 571.
  \url{http://www.gamlss.com/wp-content/uploads/2015/07/FlexibleRegressionAndSmoothingDraft-1.pdf}
\item
  Scandroglio, Giacomo \& Gori, Andrea \& Vaccaro, Emiliano \&
  Voudouris, Vlasios. (2013). Estimating VaR and ES of the spot price of
  oil using futures-varying centiles. Int. J. of Financial Engineering
  and Risk Management. 1. 6 - 19. 10.1504/IJFERM.2013.053713.
  \url{https://www.researchgate.net/publication/264815334_Estimating_VaR_and_ES_of_the_spot_price_of_oil_using_futures-varying_centiles}
\item
  RDRR.io. oil: The oil price data.
  \url{https://rdrr.io/cran/gamlss.data/man/oil.html}
\item
  Kaggle: Gabriel Idalino. GAMLSS in R - Oil Price Prediction.
  \url{https://www.kaggle.com/gabrieloliveirasan/gamlss-in-r-oil-price-prediction?scriptVersionId=74463252}
\item
  RDocumentation. gamlss.family: Family Objects for fitting a GAMLSS
  model.
  \url{https://www.rdocumentation.org/packages/gamlss.dist/versions/5.3-2/topics/gamlss.family}
\item
  Penn State - Eberly College of Science. Normal Probability Plot of
  Residuals. \url{https://online.stat.psu.edu/stat462/node/122/}
\end{itemize}

\begin{center}\rule{0.5\linewidth}{0.5pt}\end{center}

\end{document}
